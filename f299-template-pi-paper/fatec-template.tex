%%%% fatec-article.tex, 2024/03/10

%% Classe de documento
\documentclass[
  a4paper,%% Tamanho de papel: a4paper, letterpaper (^), etc.
  12pt,%% Tamanho de fonte: 10pt (^), 11pt, 12pt, etc.
  english,%% Idioma secundário (penúltimo) (>)
  brazilian,%% Idioma primário (último) (>)
]{article}

%% Pacotes utilizados
\usepackage[]{fatec-article}
\Author{1}{Name={Kuzinor. J\\ Cesar. P \\ Souza. V}}

\Author{2}{Name={\{ joao.lima135 @fatec.sp.gov.br \}\\ \{ paulo.candiani@fatec.sp.gov.br \} \\ \{ vinicius.ramos31 @fatec.sp.gov.br\} }}

%% Definição das palavras-chaves/keywords
\Keyword{1}{Manejo de bubalinos}{Buffalo management}
\Keyword{2}{Eficiência reprodutiva}{Reproductive efficiency}
\Keyword{3}{Análise de comportamento}{Behavior analysis}
\Keyword{4}{Sistema de gerenciamento}{management system}
\Keyword{5}{Bufalo}{Buffalo}

%%%% Resumo no idioma primário (brazilian)
\begin{Abstract}[brazilian]%% Idioma (brazilian ou english)
  O projeto proposto visa simplificar o gerenciamento de bubalinos, permitindo o cadastro detalhado de cada animal para a coleta e o armazenamento de dados essenciais. Incorporando recursos de inteligência artificial, o sistema detecta o período de cio das búfalas, melhorando a eficiência reprodutiva e fornecendo insights para otimizar estratégias de reprodução. Análises contínuas aprimoram a tomada de decisão, promovendo práticas sustentáveis no manejo do rebanho. Estudos utilizados como base identificaram comportamentos bovinos com técnicas de inteligência artificial, demonstrando a utilidade dessa abordagem na identificação e correlação dos comportamentos com o ambiente. Além disso, pesquisas citadas ao longo do artigo focam na melhoria genética dos bubalinos, visando aumentar a produtividade e a qualidade da carne e do leite produzidos. Com base nessas pesquisas, o projeto visa criar uma solução inovadora e eficiente para o manejo de bubalinos, unindo tecnologia de ponta com práticas agropecuárias sustentáveis.
\end{Abstract}

%%%% Resumo no idioma secundário (english)
\begin{Abstract}[english]%% Idioma (brazilian ou english)
  The proposed system aims to simplify the management of buffaloes by allowing detailed registration of each animal for the collection and storage of essential data. Incorporating artificial intelligence resources, the system detects the estrus period of buffaloes, improving reproductive efficiency and providing insights to optimize breeding strategies. Continuous analyses enhance decision-making, promoting sustainable practices in herd management. Studies used as a basis have identified bovine behaviors using artificial intelligence techniques, demonstrating the usefulness of this approach in identifying and correlating behaviors with the environment. Furthermore, research cited throughout the article focuses on improving the genetics of buffaloes to increase the productivity and quality of meat and milk produced. Based on these studies, the project aims to create an innovative and efficient solution for buffalo management, integrating cutting-edge technology with sustainable agricultural practices.
\end{Abstract}

%% Processamento de entradas (itens) do índice remissivo (makeindex)
\makeindex%

%% Arquivo(s) de referências
\addbibresource{fatec-article.bib}

%% Início do documento
\begin{document}

% Seções e subseções
%\section{Título de Seção Primária}%

%\subsection{Título de Seção Secundária}%

%\subsubsection{Título de Seção Terciária}%

%\paragraph{Título de seção quaternária}%

%\subparagraph{Título de seção quinária}%

\section*{Introdução}%
\label{sect:intro}
A primeira introdução de búfalos no Brasil teria ocorrido por volta de 1895, trazidos por condenados foragidos da Guiana Francesa em um barco que aportou na costa norte da Ilha do Marajó. A introdução mais documentada, no entanto, ocorreu por volta de 1902, com uma importação feita por Bertino Lobato de Miranda para sua Fazenda São Joaquim, localizada às margens do rio Ararí, também na Ilha do Marajó. Esses búfalos eram pretos, de procedência italiana \cite{ABCB2016}. Com o passar dos anos, diversas outras importações foram realizadas por criadores do Marajó, do Baixo Amazonas, do Nordeste, do Sul e de Minas Gerais. Atualmente, o rebanho bubalino no Brasil é de aproximadamente 1.672.956 cabeças \cite{IBGE2023}, distribuídas entre diversos estados.

A criação de búfalos é de grande importância para o atendimento da demanda alimentar (leite e carne) e também para a economia, tanto no Brasil quanto no mundo. Esses animais apresentam vantagens em relação a outros ruminantes domésticos, principalmente no que diz respeito à rusticidade, à capacidade de aproveitamento de alimentos de baixa qualidade, à adaptação a terrenos alagadiços e às variadas condições climáticas e de manejo \cite{EMBRAPA2019}. As búfalas são consideradas excelentes produtoras de leite, elas podem atingir médias superiores a 7 litros de leite por fêmea/dia, durante lactações de aproximadamente 270 dias. No entanto, a média nacional não ultrapassa 5 litros por fêmea/dia, em lactações de cerca de 250 dias \cite{Embrapa1998}. Para aumentar essa produtividade, práticas como a seleção de matrizes (definindo um mínimo produtivo para permanência no rebanho), a seleção de reprodutores (com foco em valor genético para produção leiteira), o manejo adequado e os cuidados sanitários são essenciais.

Apesar da relevância da produção de leite bubalino, ainda não existem sistemas digitais específicos para o manejo desses animais. As soluções atualmente disponíveis são direcionadas ao gado bovino, o que pode gerar inconsistências no controle da reprodução, lactação e produtividade, devido às particularidades fisiológicas dos búfalos. Muitos produtores ainda utilizam planilhas eletrônicas, tornando a consulta de dados históricos e o registro de novas informações um processo massivo, complexo e suscetível a erros devido ao grande volume de dados. Outros dependem de anotações em papel, que podem se perder com a degradação física e dificultam a consulta rápida e confiável. Essa lacuna tecnológica compromete a organização das informações, a análise de desempenho e a tomada de decisão estratégica pelos produtores.

Com o objetivo de compreender como é realizado o manejo em fazendas voltadas à produção de leite de búfala, foram realizadas visitas a propriedades localizadas na região do Vale do Ribeira. Por meio dessas visitas, foi possível conduzir a pesquisa técnica, adotando-se a metodologia qualitativa, a qual permitiu entender as particularidades de diferentes realidades produtivas. Foram entrevistados dois produtores com perfis distintos, um com um rebanho de 12 cabeças e outro com mais de 400, além de um médico-veterinário especializado, atuante em uma indústria de laticínios da região. Esse profissional presta atendimento a diversos fornecedores da empresa, oferecendo uma visão ampla sobre os padrões e cuidados adotados na criação de bubalinos.

A pesquisa com os produtores revelou que ambos não utilizam softwares específicos para o manejo de bubalinos. Ao ser feita a pergunta: “Existe algum software específico utilizado para o gerenciamento do manejo de fazenda com foco na lactação de búfalos? Se sim, qual?”, ambos responderam que desconhecem a existência de um sistema específico para o manejo de bubalinos. Eles mencionaram conhecer plataformas desenvolvidas para o manejo de bovinos, o que não se mostra totalmente efetivo, pois, quando se trata de informações reprodutivas, o tempo de gestação do bovino é de cerca de 9 meses (SILVA, 2020), enquanto o dos bubalinos é de aproximadamente 10 meses (EMBRAPA, 2007). Dessa forma, há uma diferença média de cerca de 30 dias, o que faz com que softwares com valores pré-definidos acusassem que as búfalas estão com a reprodução atrasada.

Seguindo a mesma linha de investigação, foi questionado o interesse em utilizar um sistema específico para auxiliar o manejo: “Se existisse um software específico para o gerenciamento do manejo de fazendas com foco na lactação de búfalos, que possuísse funcionalidades para identificar os búfalos com desempenho abaixo da média e para acompanhar as informações sanitárias, zootécnicas e de lactação, como ele poderia impactar a gestão e melhorar os resultados da propriedade?”. Ambos demonstraram interesse em uma solução voltada ao setor, reconhecendo que a adoção de um sistema digital poderia tornar a avaliação do desempenho da propriedade mais precisa e reduzir perdas de informações importantes.

Durante as entrevistas, também foi feita a pergunta: “Na sua opinião, quais funcionalidades seriam necessárias para que um sistema atendesse à sua forma de trabalho?”. O objetivo foi identificar possíveis lacunas na proposta atual do sistema, permitindo o planejamento de futuras atualizações. As respostas indicaram duas sugestões relevantes: (1) a implementação de uma visualização da árvore genealógica dos animais, considerada essencial para o controle das matrizes presentes na propriedade, visando sempre as que mais produzem leite; (2) uma funcionalidade que possibilite identificar rapidamente o animal por meio do celular, exibindo seu prontuário com todas as informações disponíveis, sendo que esta última já estava prevista na proposta do sistema em desenvolvimento.

Além das propriedades voltadas à produção de leite, também foi realizada uma entrevista com representantes de um Instituto de Zootecnia localizado na mesma região. Durante a entrevista, foi possível observar que o controle do rebanho, no instituto, é realizado por meio de planilhas eletrônicas separadas por áreas temáticas, como pesagem dos animais, produção de leite, registros de cruzamentos, tratamentos e separação de grupos. Cada planilha contém múltiplas abas correspondentes aos anos de registro, exigindo a repetição manual de informações entre diferentes arquivos e períodos. Essa estrutura fragmentada torna o processo massivo e suscetível a erros, além de dificultar a manutenção de um histórico consolidado dos dados do rebanho.

Este projeto está alinhado às áreas temáticas definidas pelo Fórum de Pró-Reitores de Extensão das Universidades Públicas Brasileiras (FORPROEX), atendendo aos eixos de Meio Ambiente (5) e Tecnologia (7). Também contempla as linhas de extensão voltadas ao Desenvolvimento de Produtos (7), Desenvolvimento Tecnológico (10), Gestão do Trabalho (22) e Saúde Animal (43). Tal enquadramento reforça o compromisso da proposta com os princípios da extensão universitária, contribuindo para a integração entre conhecimento científico, demandas sociais e inovação prática no setor agropecuário \cite{FORPROEX}.

\section*{OBJETIVO} \label{sect:obj}

O presente projeto tem como objetivo desenvolver uma aplicação multiplataforma (web e mobile) voltada à otimização e ao apoio da gestão do manejo de bubalinos em propriedades produtoras de leite. A ferramenta buscará centralizar informações zootécnicas, sanitárias, reprodutivas e produtivas, oferecendo ao produtor uma visão completa e integrada do desempenho do rebanho. Dessa forma, espera-se contribuir para a melhoria da eficiência produtiva, a redução de perdas de informação e o fortalecimento das estratégias de seleção, reprodução e comercialização do leite bubalino.

\subsection{Objetivos Específicos}
O sistema proposto visa atender às necessidades dos produtores por meio da implementação de funcionalidades que possibilitem o acompanhamento individualizado de cada animal e o gerenciamento de indicadores-chave de desempenho da fazenda. Entre as principais funções previstas, destacam-se:
\begin{enumerate}
    \item \textbf{Registrar dados zootécnicos},permitir o armazenamento das características e métricas corporais dos animais, facilitando o monitoramento do desenvolvimento do rebanho;
    \item \textbf{Controlar informações sanitárias}, registrar vacinas, vermífugos, medicamentos e tratamentos realizados, contribuindo para a rastreabilidade e saúde dos animais;
    \item \textbf{Gerenciar o processo reprodutivo}, acompanhar os cruzamentos realizados, gestações e partos, permitindo uma visão detalhada do ciclo reprodutivo de cada búfala;
    \item \textbf{Monitorar a lactação}, registrar e acompanhar a produção de leite de cada animal durante o ciclo produtivo, possibilitando a análise de desempenho e produtividade;
    \item \textbf{Gestão produtiva da propriedade}, controlar dados relacionados à venda de leite para a indústria, como volumes aprovados e reprovados, além de informações sobre manejo, como a distribuição de piquetes e confinamentos.
\end{enumerate}

Essas funcionalidades, integradas em uma única plataforma digital, visam fornecer ao produtor informações precisas e acessíveis, facilitando a tomada de decisões estratégicas e promovendo a modernização da gestão das propriedades bubalinas.

\section*{ESTADO DA ARTE} \label{sect:estadoarte}

Com o intuito de apoiar o desenvolvimento deste projeto, foi realizada uma pesquisa sobre trabalhos e projetos correlatos. Observou-se que muitos estudos focam na fase de lactação dos bovinos, tema que, embora semelhante ao abordado neste trabalho, apresenta algumas diferenças em sua abordagem e implementação. Ainda assim, esses projetos fornecem uma base valiosa, uma vez que reforçam a necessidade da aplicação de sistemas informatizados no setor pecuário. A adoção da tecnologia se mostra uma excelente alternativa às práticas ainda comuns em muitas propriedades, como o uso de anotações físicas e planilhas em softwares genéricos. Dessa forma, a literatura existente valida a proposta deste sistema, ao destacar os benefícios da informatização no manejo leiteiro.

O primeiro estudo analisado foi apresentado por \cite{SCHAFFER2021}, que desenvolveu uma plataforma web para o gerenciamento de bovinos, com o intuito de substituir métodos tradicionais utilizados pelos produtores rurais, como planilhas e registros em papel. O sistema foi implementado utilizando HTML, CSS, JavaScript, PHP e a biblioteca Bootstrap, além de empregar o banco de dados MySQL por meio do pacote XAMPP, disponibilizando uma interface responsiva que pode ser acessada também por dispositivos móveis. Entre suas principais funcionalidades, destacam-se o controle detalhado das informações zootécnicas dos animais, incluindo histórico de saúde, dados reprodutivos para fêmeas, e atributos cadastrais como sexo, raça e data de nascimento. O sistema permite ainda a atualização e remoção de registros individualmente, bem como a visualização do histórico sanitário de cada bovino, promovendo maior organização e segurança no gerenciamento do rebanho. Apesar disso, o projeto apresenta limitações, principalmente no que diz respeito à indisponibilidade de funcionamento em modo offline, impossibilitando seu uso em ambientes rurais sem conectividade a uma rede de internet. Além disso, o artigo destaca que melhorias futuras poderiam incluir a adoção de um servidor de banco de dados mais robusto, visando ampliar a escalabilidade, segurança e estabilidade da solução em cenários reais de produção.

O segundo projeto analisado foi o Trabalho de Conclusão de Curso desenvolvido por \cite{Pamella2017}, que consistiu na criação de um sistema voltado ao controle de rebanho bovino leiteiro. O objetivo principal foi oferecer uma solução tecnológica alinhada às reais necessidades do campo, prezando por praticidade, agilidade e eficiência no manejo dos animais. O sistema foi implementado utilizando a linguagem de programação Java, com desenvolvimento realizado na IDE NetBeans e com armazenamento de dados em um banco relacional SQL Server. Entre suas funcionalidades, destacam-se o controle reprodutivo, a pesagem de leite, a visualização dinâmica de dados e a geração de relatórios. A aplicação foi testada na fazenda Dois Irmãos, onde foram observados resultados significativos. O controle reprodutivo dos bovinos tornou-se mais organizado e eficiente, o processo de pesagem de leite foi automatizado, reduzindo erros comuns no método manual, e as necessidades específicas do proprietário, sobretudo no que se refere à geração de relatórios, foram plenamente atendidas. Além disso, a acessibilidade e a variedade de formas de visualização dos dados facilitaram a administração da propriedade e contribuíram para a tomada de decisões estratégicas. Os resultados obtidos evidenciam o potencial da tecnologia como ferramenta essencial para o avanço da pecuária leiteira.

O terceiro estudo relevante é apresentado por \cite{LIMA2023}, que propôs o desenvolvimento de um web aplicativo denominado \textit{Leite Cowtrol}, voltado ao controle da produção leiteira e gestão reprodutiva de gado leiteiro. O trabalho surgiu a partir da identificação de limitações presentes em soluções já existentes no mercado, como interfaces pouco intuitivas, inconsistências em registros de inseminação e a ausência de ferramentas que auxiliem diretamente na tomada de decisão do produtor. Para sua implementação, foram utilizadas tecnologias como HTML5, CSS3, JavaScript e o framework React.js, escolhidos por sua compatibilidade com dispositivos móveis, responsividade e ecossistema consolidado. O armazenamento de dados foi realizado em MySQL, complementado pelo uso de \textit{LocalStorage} para manter persistência local e personalização na experiência do usuário. Entre as funcionalidades propostas, destacam-se o gerenciamento do ciclo de vida dos animais e o monitoramento da produção de leite, recursos fundamentais para melhorar a organização do rebanho e otimizar a produtividade. Embora o protótipo tenha demonstrado potencial para atender aos requisitos definidos e contribuir para o aprimoramento dos envolvidos no projeto, o estudo apresenta limitações importantes, como a ausência de testes práticos em propriedades rurais, impossibilitando a validação de sua eficácia em cenários reais e restringindo a análise de desempenho em condições operacionais.

Observa-se que ambos os trabalhos analisados possuem como foco exclusivo a gestão de rebanhos bovinos leiteiros, não contemplando particularidades de outras espécies, como bubalinos, que apresentam características produtivas, sanitárias e reprodutivas distintas. Além disso, as soluções propostas ainda demonstram limitações consideráveis no uso prático, como a dependência de conectividade constante, o predomínio de aplicações web ou desktop que restringem a mobilidade e dificultam o registro de informações em campo, bem como a ausência de validação em propriedades rurais reais, o que compromete a análise de desempenho sob condições operacionais. Outro aspecto observado é a falta de discussão sobre hospedagem e infraestrutura de implantação, não abordando custos operacionais, escalabilidade dos sistemas ou requisitos de disponibilidade.  Também se destaca a inexistência de mecanismos baseados em inteligência artificial capazes de apoiar a tomada de decisões estratégicas, detecção precoce de problemas sanitários ou geração de previsões produtivas. Nesse contexto, o presente projeto busca suprir essas lacunas ao propor uma solução híbrida composta por plataforma web e aplicativo mobile, priorizando mobilidade, acessibilidade e autonomia do usuário. Aliada a isso, a integração de modelos de aprendizado de máquina constitui um diferencial técnico relevante, permitindo a emissão de alertas sanitários e predições de produção leiteira, ampliando o suporte ao manejo e favorecendo decisões fundamentadas em evidências.

\section*{METODOLOGIA} \label{sect:metodologia}

O sistema proposto contempla o desenvolvimento de uma aplicação multiplataforma, composta por versões para web e dispositivos móveis, que atua como o principal meio de interação entre o usuário e os dados da propriedade. A aplicação oferece uma interface gráfica intuitiva, permitindo o acompanhamento completo das informações cadastradas e garantindo um controle preciso e em tempo real do rebanho, desde que a plataforma seja constantemente alimentada pelo usuário. Cada versão da aplicação desempenha um papel específico: a plataforma mobile, direcionada a funcionários e veterinários, prioriza a agilidade no lançamento de informações em campo; enquanto a versão web, voltada a gestores e proprietários, proporciona uma visualização mais ampla e detalhada dos dados em ambiente de escritório. Ambas atuam de forma integrada, promovendo sincronização automática e assegurando a consistência das informações em tempo real.

No \Cref{fcht:fluxograma1} é possivel verificar o fluxo que o Funcionário/Veterinário vão seguir, ao acessa o sistema por meio do aplicativo móvel, por onde realiza a inserção e consulta de informações zootécnicas, sanitárias, reprodutivas e produtivas. Quando ocorre o registro de uma nova ordenha, o sistema aciona uma inteligência artificial desenvolvida com base no modelo Gemini, responsável por identificar possíveis ocorrências de mastite nos animais. Caso seja detectada alguma anomalia, a IA realiza automaticamente a classificação do nível de gravidade da infecção, gerando alertas que auxiliam na execução de medidas corretivas e preventivas. As inserções de dados realizadas pelo aplicativo são enviadas à API, que as processa e atualiza no banco de dados principal. Esses registros alimentam também a segunda inteligência artificial do projeto, baseada no algoritmo Random Forest Regressor, utilizada para realizar predições da produção leiteira de cada animal.

\begin{flowchart}[!htb]
\centering
\caption{Fluxograma visão do Funcionário/Veterinário}%
\label{fcht:fluxograma1}
\includegraphics[scale=0.27]{diagrama_funcionario}
\SourceOrNote{Autoria Própria (2025)}
\end{flowchart}

\newpage

Já no \Cref{fcht:fluxograma2}, o usuário acessa a plataforma web, onde são disponibilizados painéis interativos que apresentam indicadores consolidados de produtividade, sanidade e distribuição do rebanho. Nessa interface, o gestor pode consultar o prontuário individual dos animais, visualizar as predições geradas pela IA de regressão e acompanhar os alertas oriundos da análise de mastite. Caso necessário, é possível inserir ou atualizar informações diretamente pela plataforma web, mantendo a consistência do banco de dados sem a necessidade de acesso ao aplicativo móvel. Essa estrutura integrada permite que o gestor acompanhe em tempo real o desempenho produtivo da propriedade, utilizando dados concretos e análises automatizadas para fundamentar suas decisões de manejo e de investimento.

\begin{flowchart}[!htb]
\centering
\caption{Fluxograma visão do Proprietário/Gestor}%
\label{fcht:fluxograma2}
\includegraphics[scale=0.28]{diagrama_propreitario}
\SourceOrNote{Autoria Própria (2025)}
\end{flowchart}

Foi apresentada, anteriormente, toda a fundamentação teórica referente ao fluxo de funcionamento do sistema. Para que esse fluxo ocorra de forma adequada, é necessário o uso de um conjunto de ferramentas e tecnologias que viabilizam a comunicação entre as diferentes camadas da aplicação, conforme ilustrado na \Cref{fcht:arquitetura}. A interação com o usuário inicia-se nas etapas I e II. A Etapa I corresponde à plataforma web, desenvolvida em \textit{React} com o framework \textit{Next.js}. Atualmente, essa plataforma encontra-se hospedada na Vercel, uma solução \textit{serverless} que dispensa a necessidade de um servidor dedicado. A Etapa II refere-se à plataforma mobile, desenvolvida em \textit{React Native}, framework que consiste em um conjunto de ferramentas voltadas à criação de aplicações móveis nativas \cite{Bruna2021}. O aplicativo tem como principal funcionalidade oferecer uma interface simplificada e intuitiva, voltada ao lançamento de dados em campo.

Ambas as plataformas comunicam-se com o Back-end (etapa III) por meio de uma Interface de Programação de Aplicações, do inglês \textit{Application Programming Interface} (API), desenvolvida em \textit{TypeScript} com o framework \textit{Nest.js}. Como parte do Back-end, foram implementados recursos adicionais com o objetivo de aperfeiçoar a usabilidade e a fluidez do sistema, destacando-se a integração com dois modelos de Inteligência Artificial(IA). A primeira delas é a API \textit{Gemini} (etapa IV), um modelo generativo multimodal desenvolvido pelo Google. No contexto deste projeto, a \textit{Gemini} é utilizada no sistema de alertas, por meio do modelo \textit{gemini-1.5-flash}. Essa implementação analisa um \textit{prompt} elaborado especificamente para o domínio do sistema e define a prioridade de cada alerta como alta, média ou baixa, simplificando de maneira significativa a lógica de priorização e resposta.

A segunda IA (etapa V) foi desenvolvida pelo próprio grupo e baseia-se no algoritmo \textit{Random Forest Regressor}, um método de aprendizado supervisionado que combina múltiplas árvores de decisão para realizar previsões numéricas. O modelo é treinado com diversas amostras aleatórias dos dados e, para cada árvore, um subconjunto de atributos, conhecidos como \textit{features}, é selecionado de forma aleatória, promovendo diversidade entre os estimadores. A predição final é obtida a partir da média dos resultados de todas as árvores, aumentando a precisão e reduzindo o risco de sobreajuste, conhecido como \textit{overfitting}, em comparação com o uso de uma única árvore de decisão. No contexto deste trabalho, o modelo tem como objetivo prever individualmente a produção de leite de cada fêmea, utilizando seu histórico produtivo, reprodutivo, sanitário, zootécnico e genético. Essas informações são consolidadas em um conjunto de \textit{features} estruturadas e, a partir delas, o sistema gera uma predição que classifica o potencial produtivo em categorias, sendo elas alto, bom, médio ou baixo. O resultado é ainda comparado com a média da propriedade e fornece informações adicionais, como o volume previsto em litros, o percentual em relação à média e a data da predição.

Para disponibilizar todo o Back-end de forma operacional e acessível, o mesmo foi hospedado na nuvem utilizando o serviço \textit{Amazon EC2}, \textit{Elastic Compute Cloud}, da \textit{Amazon Web Services} (AWS). Essa escolha permite escalar a infraestrutura dinamicamente, otimizando custos e garantindo melhor desempenho do sistema. A API é responsável por gerenciar toda a lógica de negócios, incluindo a criação, consulta, atualização e exclusão de registros no banco de dados, além de processar requisições provenientes das plataformas web e mobile. O controle de \textit{deploy} automático (etapa VI) é realizado por meio do \textit{GitHub Actions}, que integra diretamente com a instância EC2, automatizando atualizações e correções e assegurando que a aplicação permaneça sempre disponível e atualizada para os usuários.

Para o armazenamento das informações, o banco de dados do sistema (etapa VII) foi desenvolvido em \textit{PostgreSQL}, escolhido pela sua robustez e pela consistência oferecida pelo modelo relacional na organização e integridade dos dados. Essa estrutura permite armazenar grandes volumes de informações de forma estruturada, garantindo que registros relacionados, como dados de produção, lactação e histórico sanitário dos animais, sejam mantidos de forma segura e coerente. Atualmente, o banco de dados encontra-se hospedado na nuvem por meio da plataforma \textit{Supabase}, que oferece compatibilidade nativa com o PostgreSQL, além de recursos adicionais de autenticação e armazenamento.

\begin{flowchart}[!htb]
\centering
\caption{Arquitetura do fluxo das redes}%
\label{fcht:arquitetura}
\includegraphics[scale=0.15]{buffs-arch-model}
\SourceOrNote{Autoria Própria (2025)}
\end{flowchart}

Após a análise do fluxo apresentado, nota-se que grande parte do sistema encontra-se hospedada na nuvem, seja por meio de serviços \textit{serverless}, bancos de dados em ambiente \textit{cloud} ou máquinas virtuais dedicadas. Essas escolhas, no momento atual, mostraram-se adequadas sob a ótica do grupo, uma vez que permitem distribuir as \textit{stacks} entre diferentes serviços, evitando a sobrecarga de um único sistema. Além disso, essa configuração possibilita a realização de testes de forma gratuita, sem a necessidade de investimentos iniciais em infraestrutura. Outro fator determinante nas decisões de arquitetura foi o alinhamento com os conteúdos abordados em sala de aula, privilegiando tecnologias e metodologias com as quais o grupo possui maior afinidade e embasamento técnico. Essa estratégia busca não apenas consolidar o aprendizado adquirido ao longo do curso, mas também garantir que o desenvolvimento ocorra de maneira segura e fundamentada em conhecimentos previamente assimilados.



\section*{RESULTADOS PRELIMINARES} \label{sect:resultados}

Para avaliar a eficácia do modelo de IA desenvolvido para o sistema, foram realizadas análises comparativas entre as predições geradas e os valores reais de produção de leite. Com o desenvolvimento da IA baseada no modelo Random Forest Regressor, o desempenho do sistema foi avaliado por meio do coeficiente de determinação (\(R^2\)), obtendo-se o valor de 0,73. Como mostrado no Gráfico \Cref{grph:example}, a dispersão das predições em relação aos valores reais indica que a maior parte das estimativas se aproxima da diagonal \(y = x\), evidenciando que o modelo consegue explicar 73\% da variabilidade observada na produção de leite. Apesar de alguns desvios individuais, os resultados demonstram que o sistema fornece previsões consistentes e confiáveis para o manejo do rebanho.  O modelo considera informações como dias em lactação, produção média de lactação, histórico da produção média, idade da búfala, idade no primeiro parto e intervalo entre partos, permitindo análises detalhadas e suporte à tomada de decisão baseada em dados.

\begin{graph}[!h]
\centering
\SetCaptionWidth{\ifbool{@LayoutA}{0.47}{0.49}\linewidth}
\caption{Dispersão das predições do modelo}%
\label{grph:example}
\includegraphics[width = \CaptionWidth]{R}
\SourceOrNote{Autoria Própria (2025)}
\end{graph}


\section*{CONCLUSÃO} \label{sect:conclusao}

Com o propósito de contribuir para a consecução dos Objetivos de Desenvolvimento Sustentável estabelecidos pela ONU, destaca-se a relevância da oitava ODS, voltada para o crescimento econômico. Nesse contexto, aprimorar o gerenciamento dos dados relacionados aos bubalinos, incluindo informações zootécnicas, sanitárias e histórico do animal, torna-se crucial. Tal aprimoramento visa mitigar a falta de transparência em situações como a venda dos animais para potenciais clientes.
Adicionalmente, a nona ODS, centrada na inovação e infraestrutura, integra-se harmoniosamente ao escopo do projeto em questão. Este, focado na otimização do manejo de bubalinos por meio de uma aplicação móvel inovadora, destaca-se pela incorporação de tecnologias avançadas, como algoritmos de inteligência artificial. Essa abordagem inovadora visa aprimorar significativamente a eficiência no manejo, proporcionando benefícios tangíveis para a sustentabilidade econômica e ambiental, alinhando-se assim aos objetivos globais delineados pela ONU.

O sistema proposto visa proporcionar aos responsáveis pelo manejo de bubalinos uma plataforma eficaz para o cadastro e monitoramento de seus animais. Através do sistema, será possível realizar o cadastro detalhado de cada búfalo, permitindo a coleta e armazenamento dos dados essenciais em seus respectivos prontuários. Esse processo não apenas simplifica a gestão, mas também promove transparência, oferecendo aos criadores um controle mais preciso sobre as informações relacionadas aos seus rebanhos.
Além disso, o sistema incorpora recursos de inteligência artificial destinados a detectar o período de cio das búfalas. Essa funcionalidade não apenas aprimora a eficiência reprodutiva do criadouro, mas também fornece insights valiosos para otimizar estratégias de reprodução. A capacidade de identificar o momento propício para a reprodução é um avanço significativo no gerenciamento reprodutivo do rebanho.Em síntese, o sistema proposto não apenas simplifica os processos de gerenciamento de bubalinos, mas também integra tecnologias avançadas para aprimorar a eficiência reprodutiva e promover práticas sustentáveis no manejo do rebanho.

Adicionalmente, como parte essencial para futuros desenvolvimentos, identifica-se a necessidade de aprofundar a compreensão no gerenciamento do manejo, permitindo a geração automática de relatórios. Esse aprimoramento possibilitará uma análise mais abrangente e eficaz das informações coletadas, contribuindo para uma tomada de decisão mais informada por parte dos responsáveis pelo manejo de bubalinos.
Outro ponto a ser considerado em perspectiva futura é a contínua análise e aprimoramento da aplicação móvel. Isso incluirá a realização de testes práticos junto a diferentes criadouros, a fim de validar a eficácia da aplicação em variados cenários de manejo. A condução desses testes proporcionará valiosos insights para ajustes e refinamentos contínuos, garantindo que a aplicação atenda plenamente às necessidades e nuances específicas de diversos criadouros. Esse ciclo de aprimoramento contínuo assegura a adaptabilidade e relevância da solução ao longo do tempo.

Em síntese, este estudo destaca a significativa relevância comercial e o potencial benéfico da plataforma para o manejo de bubalinos. Embora haja a necessidade de realização de testes e análises completas para avaliar a viabilidade do projeto, seu embasamento científico sólido e o potencial para aprimorar o gerenciamento do manejo de bubalinos o qualificam como uma iniciativa promissora. Esta plataforma mostra-se capaz de contribuir para o avanço do conhecimento e o desenvolvimento de soluções efetivas no contexto do manejo sustentável de rebanhos bubalinos.


\printbibliography

%% Elementos pós-textuais (opcionais): Apêndice e Anexo
%Caso for utilizar, basta retirar o símbolo de % na frente do comando
%\input{./Extras/post-textual}

%% Fim do documento
\end{document}