O presente projeto tem como objetivo desenvolver uma aplicação multiplataforma (web e mobile) voltada à otimização e ao apoio da gestão do manejo de bubalinos em propriedades produtoras de leite. A ferramenta buscará centralizar informações zootécnicas, sanitárias, reprodutivas e produtivas, oferecendo ao produtor uma visão completa e integrada do desempenho do rebanho. Dessa forma, espera-se contribuir para a melhoria da eficiência produtiva, a redução de perdas de informação e o fortalecimento das estratégias de seleção, reprodução e comercialização do leite bubalino.

\subsection{Objetivos Específicos}
O sistema proposto visa atender às necessidades dos produtores por meio da implementação de funcionalidades que possibilitem o acompanhamento individualizado de cada animal e o gerenciamento de indicadores-chave de desempenho da fazenda. Entre as principais funções previstas, destacam-se:
\begin{enumerate}
    \item \textbf{Registrar dados zootécnicos},permitir o armazenamento das características e métricas corporais dos animais, facilitando o monitoramento do desenvolvimento do rebanho;
    \item \textbf{Controlar informações sanitárias}, registrar vacinas, vermífugos, medicamentos e tratamentos realizados, contribuindo para a rastreabilidade e saúde dos animais;
    \item \textbf{Gerenciar o processo reprodutivo}, acompanhar os cruzamentos realizados, gestações e partos, permitindo uma visão detalhada do ciclo reprodutivo de cada búfala;
    \item \textbf{Monitorar a lactação}, registrar e acompanhar a produção de leite de cada animal durante o ciclo produtivo, possibilitando a análise de desempenho e produtividade;
    \item \textbf{Gestão produtiva da propriedade}, controlar dados relacionados à venda de leite para a indústria, como volumes aprovados e reprovados, além de informações sobre manejo, como a distribuição de piquetes e confinamentos.
\end{enumerate}

Essas funcionalidades, integradas em uma única plataforma digital, visam fornecer ao produtor informações precisas e acessíveis, facilitando a tomada de decisões estratégicas e promovendo a modernização da gestão das propriedades bubalinas.