Desenvolver uma aplicação multiplataforma voltada à otimização e ao apoio da gestão de bubalinos em propriedades produtoras de leite, centralizando informações zootécnicas, sanitárias, reprodutivas e produtivas, e incorporando técnicas de Inteligência Artificial para previsão da produção de leite e geração de alertas automatizados. A plataforma fornecerá ao produtor uma visão integrada do desempenho e saúde do rebanho, contribuindo para a eficiência produtiva, redução de perdas de informação e aprimoramento das estratégias de seleção, reprodução e comercialização do leite bubalino.

Para atender às necessidades dos produtores, o sistema implementará funcionalidades que permitam o acompanhamento individualizado de cada animal e o gerenciamento de indicadores-chave de desempenho da fazenda. Entre as principais funções previstas, destacam-se:

\begin{enumerate}
    \item Registrar dados zootécnicos, métricas corporais dos animais, permitindo acompanhamento individual e monitoramento do desenvolvimento do rebanho.
    \item Controlar informações sanitárias, incluindo vacinas, medicamentos e tratamentos, garantindo rastreabilidade e saúde do rebanho.
    \item Gerir o ciclo reprodutivo, acompanhando cruzamentos, gestações e partos.
    \item Monitorar a produção de leite, registrando volumes por animal durante o ciclo produtivo, para análise de desempenho.
    \item Implementar recursos de Inteligência Artificial para prever a produção individual de leite das fêmeas, classificando o potencial produtivo e gerando alertas inteligentes para suporte à tomada de decisão.
    \item Gerir a produção e comercialização do leite, controlando volumes vendidos, quantidades aprovadas e reprovadas, além de informações sobre manejo, como distribuição de piquetes e confinamentos.
\end{enumerate}


Essas funcionalidades, integradas em uma única plataforma digital, visam fornecer ao produtor informações precisas e acessíveis, facilitando a tomada de decisões estratégicas e promovendo a modernização da gestão das propriedades que visão a bubalinocultura.