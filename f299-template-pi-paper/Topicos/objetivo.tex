O objetivo desse projeto é elaborar um sistema que ajude na gestão do manejo de bubalinos com o controle de dados zootécnicos e sanitários, além de informações reprodutivas. O sistema ainda contará com a ajuda da IA para um melhor controle dos dados reprodutivos. Isso incluirá a supervisão do período de cio, durante o qual a fêmea está sexualmente receptiva e pronta para a reprodução. Dessa forma, o criador poderá otimizar a reprodução de seu rebanho permitindo a inseminação artificial ou a monta natural no momento certo.
 O projeto ainda tem como objetivo: 

\begin{enumerate}
    \item Efetuar uma analise dos dados zootécnicos e sanitarios necessarios de cada bubalino;
    \item Identificar problemas que tenham potencial para se propagar pelo manejo, para que o proprietário possa adotar medidas preventivas;
    \item Analisar o comportamento e sinais físicos das fêmeas bubalinas, referente ao período de cio;
    \item Após a análise e identificação dos comportamentos, será implementada a inteligência artificial para tornar o processo mais efetivo e automatizado.
\end{enumerate}
