Com o intuito de apoiar o desenvolvimento deste projeto, foi realizada uma pesquisa sobre trabalhos e projetos correlatos. Observou-se que muitos estudos focam na fase de lactação dos bovinos, tema que, embora semelhante ao abordado neste trabalho, apresenta algumas diferenças em sua abordagem e implementação. Ainda assim, esses projetos fornecem uma base valiosa, uma vez que reforçam a necessidade da aplicação de sistemas informatizados no setor pecuário. A adoção da tecnologia se mostra uma excelente alternativa às práticas ainda comuns em muitas propriedades, como o uso de anotações físicas e planilhas em softwares genéricos. Dessa forma, a literatura existente valida a proposta deste sistema, ao destacar os benefícios da informatização no manejo leiteiro.

O primeiro estudo analisado foi apresentado por \cite{SCHAFFER2021}, que desenvolveu uma plataforma web para o gerenciamento de bovinos, com o intuito de substituir métodos tradicionais utilizados pelos produtores rurais, como planilhas e registros em papel. O sistema foi implementado utilizando HTML, CSS, JavaScript, PHP e a biblioteca Bootstrap, além de empregar o banco de dados MySQL por meio do pacote XAMPP, disponibilizando uma interface responsiva que pode ser acessada também por dispositivos móveis. Entre suas principais funcionalidades, destacam-se o controle detalhado das informações zootécnicas dos animais, incluindo histórico de saúde, dados reprodutivos para fêmeas, e atributos cadastrais como sexo, raça e data de nascimento. O sistema permite ainda a atualização e remoção de registros individualmente, bem como a visualização do histórico sanitário de cada bovino, promovendo maior organização e segurança no gerenciamento do rebanho. Apesar disso, o projeto apresenta limitações, principalmente no que diz respeito à indisponibilidade de funcionamento em modo offline, impossibilitando seu uso em ambientes rurais sem conectividade a uma rede de internet. Além disso, o artigo destaca que melhorias futuras poderiam incluir a adoção de um servidor de banco de dados mais robusto, visando ampliar a escalabilidade, segurança e estabilidade da solução em cenários reais de produção.

O segundo projeto analisado foi o Trabalho de Conclusão de Curso desenvolvido por \cite{Pamella2017}, que consistiu na criação de um sistema voltado ao controle de rebanho bovino leiteiro. O objetivo principal foi oferecer uma solução tecnológica alinhada às reais necessidades do campo, prezando por praticidade, agilidade e eficiência no manejo dos animais. O sistema foi implementado utilizando a linguagem de programação Java, com desenvolvimento realizado na IDE NetBeans e com armazenamento de dados em um banco relacional SQL Server. Entre suas funcionalidades, destacam-se o controle reprodutivo, a pesagem de leite, a visualização dinâmica de dados e a geração de relatórios. A aplicação foi testada na fazenda Dois Irmãos, onde foram observados resultados significativos. O controle reprodutivo dos bovinos tornou-se mais organizado e eficiente, o processo de pesagem de leite foi automatizado, reduzindo erros comuns no método manual, e as necessidades específicas do proprietário, sobretudo no que se refere à geração de relatórios, foram plenamente atendidas. Além disso, a acessibilidade e a variedade de formas de visualização dos dados facilitaram a administração da propriedade e contribuíram para a tomada de decisões estratégicas. Os resultados obtidos evidenciam o potencial da tecnologia como ferramenta essencial para o avanço da pecuária leiteira.

O terceiro estudo relevante é apresentado por \cite{LIMA2023}, que propôs o desenvolvimento de um web aplicativo denominado \textit{Leite Cowtrol}, voltado ao controle da produção leiteira e gestão reprodutiva de gado leiteiro. O trabalho surgiu a partir da identificação de limitações presentes em soluções já existentes no mercado, como interfaces pouco intuitivas, inconsistências em registros de inseminação e a ausência de ferramentas que auxiliem diretamente na tomada de decisão do produtor. Para sua implementação, foram utilizadas tecnologias como HTML5, CSS3, JavaScript e o framework React.js, escolhidos por sua compatibilidade com dispositivos móveis, responsividade e ecossistema consolidado. O armazenamento de dados foi realizado em MySQL, complementado pelo uso de \textit{LocalStorage} para manter persistência local e personalização na experiência do usuário. Entre as funcionalidades propostas, destacam-se o gerenciamento do ciclo de vida dos animais e o monitoramento da produção de leite, recursos fundamentais para melhorar a organização do rebanho e otimizar a produtividade. Embora o protótipo tenha demonstrado potencial para atender aos requisitos definidos e contribuir para o aprimoramento dos envolvidos no projeto, o estudo apresenta limitações importantes, como a ausência de testes práticos em propriedades rurais, impossibilitando a validação de sua eficácia em cenários reais e restringindo a análise de desempenho em condições operacionais.

Observa-se que ambos os trabalhos analisados possuem como foco exclusivo a gestão de rebanhos bovinos leiteiros, não contemplando particularidades de outras espécies, como bubalinos, que apresentam características produtivas, sanitárias e reprodutivas distintas. Além disso, as soluções propostas ainda demonstram limitações consideráveis no uso prático, como a dependência de conectividade constante, o predomínio de aplicações web ou desktop que restringem a mobilidade e dificultam o registro de informações em campo, bem como a ausência de validação em propriedades rurais reais, o que compromete a análise de desempenho sob condições operacionais. Outro aspecto observado é a falta de discussão sobre hospedagem e infraestrutura de implantação, não abordando custos operacionais, escalabilidade dos sistemas ou requisitos de disponibilidade.  Também se destaca a inexistência de mecanismos baseados em inteligência artificial capazes de apoiar a tomada de decisões estratégicas, detecção precoce de problemas sanitários ou geração de previsões produtivas. Nesse contexto, o presente projeto busca suprir essas lacunas ao propor uma solução híbrida composta por plataforma web e aplicativo mobile, priorizando mobilidade, acessibilidade e autonomia do usuário. Aliada a isso, a integração de modelos de aprendizado de máquina constitui um diferencial técnico relevante, permitindo a emissão de alertas sanitários e predições de produção leiteira, ampliando o suporte ao manejo e favorecendo decisões fundamentadas em evidências.