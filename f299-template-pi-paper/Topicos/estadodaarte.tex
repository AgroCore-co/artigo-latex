Nesta seção, estado da arte, ela se divide em três subseções diferentes, cada uma delas explorando aspectos importantes da pesquisa. A primeira subseção aborda estudos sobre os dados zootécnicos dos bubalinos, mostrando assim a melhora na produção. Na segunda subseção, é citado um projeto que utilizou IA para identificar comportamento de bovinos. E na última subseção, ressalta-se a importância de identificar animais com alto potencial genético para a produção de carne e leite de qualidade. Essa identificação é fundamental para selecionar reprodutores de alta qualidade, com base em critérios genéticos específicos.

 \subsection*{Estudos dos dados zootécnicos, mostrando melhoras na produção}
 
  Os dados para este estudo foram coletados mensalmente a partir de junho de 2007 em fazendas com rebanhos das raças Murrah e Mediterrâneo na região do Pará. O acompanhamento incluiu o Controle Leiteiro, seguindo as normas oficiais da ABCZ para o gir leiteiro, com adaptações à espécie. Os dados foram organizados em arquivos para manejo, avaliação genética e estudos da estrutura populacional dos bubalinos. 
  Foi realizado coletas de dados como:  pesagens das fêmeas e dos bezerros e anotadas outras informações importantes, como identificação (brinco) da fêmea e do bezerro, nome dos animais, sexo da cria, data do parto, peso do leite (kg), tempo de ordenha (pré-dipping, ordenha e pós-dipping) e observações pertinentes, relacionadas à saúde dos animais, ao seu comportamento ou manejo alimentar diferenciado das demais. Além disso, outras informações foram solicitadas aos criadores, como complemento ao arquivo de dados em formação, e armazenadas na Embrapa Amazônia Oriental, como a data de nascimento da búfala, identificação e controle de parentesco. 
 Os resultados preliminares  obtidos deste estudo indicaram que a raça Murrah apresentou maior produção de leite do que a raça Mediterrâneo, tanto em termos de média quanto de variabilidade. Além disso, foi possível identificar diferenças significativas entre as categorias de idade e estágios reprodutivos das fêmeas, com a produção de leite aumentando até o pico de lactação e diminuindo em seguida. Os dados coletados também permitiram a avaliação genética dos animais, com a identificação de touros e matrizes com maior potencial para produção de leite e outras características de interesse. \cite{Righetti.C}




\subsection*{IA para identificar comportamento de bovinos}

Neste estudo, os dados foram coletados por meio de um colar com sensores que coletam os dados de posicionamento, movimentação do animal e luminosidade do ambiente. Os comportamentos primários do animal foram aferidos utilizando os dados coletados. A combinação desses comportamentos com os dados do ambiente pode indicar outros comportamentos, chamados de comportamentos secundários. Os dados foram coletados por um período de 15 dias e uma nova coleta foi realizada para a validação dos comportamentos classificados. Para a organização dos dados, foi necessário criar um par entre a instância e a identificação de classe da instância, observando os animais e anotando qual atividade o gado está desempenhando e em qual instante isso ocorreu.
Os dados coletados neste estudo foram de posicionamento, movimentação do animal e luminosidade do ambiente. Esses dados foram coletados por meio de um colar com sensores que os registram. Os comportamentos primários do animal (Andando, Pastando, Parado Em Pé, Parado Deitado, Ruminando Em Pé e Ruminando Deitado) foram aferidos utilizando os dados coletados.O estudo obteve resultados interessantes na identificação dos comportamentos bovinos. No primeiro experimento, o algoritmo de classificação supervisionada obteve uma taxa de acerto de 70,5\% na identificação dos comportamentos primários dos animais. No segundo experimento, após a associação das classes Em Pé e Deitado em uma única, a taxa de acerto aumentou para 84,86\%. Além disso, foi possível identificar comportamentos secundários a partir da relação dos comportamentos primários com os dados do ambiente. Esses comportamentos secundários incluem Comendo, Bebendo, Descansando e Explorando. Esses resultados mostram que a utilização de técnicas de inteligência artificial pode ser útil na identificação dos comportamentos bovinos e na relação desses comportamentos com o ambiente. \cite{Fernando.L}.


\subsection*{Melhoramento Genético em Bubalinos}
O estudo, liderado pela Dra. Cintia Marcondes e sua equipe, envolveu a coleta de dados em fazendas de diversas regiões do Brasil, com a participação ativa de criadores e equipes de campo formadas por alunos de Mestrado do Curso de Ciência Animal e alunos de Agrárias da UFRA e da UFPA. O principal objetivo foi realizar análises genéticas para selecionar búfalos com potencial superior para produção de carne e leite de qualidade. Ao integrar dados de diferentes estados, como Pará, Rondônia, Rio Grande do Sul e Bahia, o estudo buscou avaliar e aprimorar as características genéticas dos rebanhos de bubalinos. Os resultados alcançados incluem a identificação de animais com potencial genético superior, a seleção criteriosa de reprodutores de alta qualidade e a consequente melhoria da produtividade e qualidade dos rebanhos. Esses avanços contribuem significativamente para o desenvolvimento de rebanhos mais produtivos e adaptados às demandas do mercado, promovendo benefícios tanto para os criadores quanto para o setor como um todo.
\cite{Ribamar.J}.