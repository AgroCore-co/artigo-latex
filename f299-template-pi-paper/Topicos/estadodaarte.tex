Com o intuito de apoiar o desenvolvimento deste projeto, foi realizada uma pesquisa sobre trabalhos e projetos correlatos. Observou-se que muitos estudos focam na fase de lactação dos bovinos, tema que, embora semelhante ao abordado neste trabalho, apresenta algumas diferenças em sua abordagem e implementação. Ainda assim, esses projetos fornecem uma base valiosa, uma vez que reforçam a necessidade da aplicação de sistemas informatizados no setor pecuário. A adoção da tecnologia se mostra uma excelente alternativa às práticas ainda comuns em muitas propriedades, como o uso de anotações físicas e planilhas em softwares genéricos. Dessa forma, a literatura existente valida a proposta deste sistema, ao destacar os benefícios da informatização no manejo leiteiro.

Um dos estudos relevantes encontrados é o de \cite{Carlos1995}, que apresenta o SISCOREB, um sistema computacional desenvolvido para o controle de rebanhos leiteiros. Entre suas funcionalidades, destacam-se o controle zootécnico, acompanhamento da produção leiteira, controle sanitário, gerenciamento reprodutivo e de manejo. O sistema foi concebido para suprir as demandas de produtores e técnicos, promovendo maior organização dos dados do rebanho e auxiliando no processo de tomada de decisões. No entanto, apesar de sua importância histórica, o SISCOREB apresenta limitações diante das tecnologias atuais, como a falta de recursos de mobilidade e a inexistência de integração com sensores inteligentes. Tais deficiências evidenciam a necessidade de soluções mais modernas, adaptadas às realidades e exigências do ambiente rural contemporâneo.

O segundo projeto analisado foi o Trabalho de Conclusão de Curso desenvolvido por \cite{Pamella2017}, que consistiu na criação de um sistema voltado ao controle de rebanho bovino leiteiro. O objetivo principal foi oferecer uma solução tecnológica alinhada às reais necessidades do campo, prezando por praticidade, agilidade e eficiência no manejo dos animais. O sistema foi implementado utilizando a linguagem de programação Java, com desenvolvimento realizado na IDE NetBeans e com armazenamento de dados em um banco relacional SQL Server. Entre suas funcionalidades, destacam-se o controle reprodutivo, a pesagem de leite, a visualização dinâmica de dados e a geração de relatórios. A aplicação foi testada na fazenda Dois Irmãos, onde foram observados resultados significativos. O controle reprodutivo dos bovinos tornou-se mais organizado e eficiente, o processo de pesagem de leite foi automatizado, reduzindo erros comuns no método manual, e as necessidades específicas do proprietário, sobretudo no que se refere à geração de relatórios, foram plenamente atendidas. Além disso, a acessibilidade e a variedade de formas de visualização dos dados facilitaram a administração da propriedade e contribuíram para a tomada de decisões estratégicas. Os resultados obtidos evidenciam o potencial da tecnologia como ferramenta essencial para o avanço da pecuária leiteira.

O terceiro trabalho analisado trata-se da dissertação de mestrado de \cite{Alexandre2008}, orientada pelo professor Gustavo Augusto de Andrade e coorientada por José Cláudio Reis, desenvolvida na Universidade de Alfenas (UNIFENAS) em 2008. O objetivo central do projeto foi desenvolver um sistema de informação capaz de auxiliar o produtor rural na tomada de decisões, utilizando dados provenientes da escrituração zootécnica e indicadores produtivos e reprodutivos do rebanho leiteiro. O sistema foi concebido para otimizar o manejo dos animais, com foco na simplicidade de uso, permitindo sua adoção mesmo por usuários com pouca familiaridade com computadores. Entre os principais resultados alcançados, destaca-se o desenvolvimento de uma interface amigável e intuitiva, o que facilitou o uso por parte dos produtores. O sistema possibilita o registro individual de cada animal, contemplando dados de controle produtivo, reprodutivo, sanitário e pesagem. Além disso, a ferramenta contribui significativamente para o acompanhamento preciso das informações do rebanho, promovendo uma tomada de decisão mais eficiente. Um diferencial importante do sistema é a dispensa de impressoras, permitindo a visualização direta dos relatórios na tela e, assim, reduzindo custos operacionais. O desenvolvimento foi realizado utilizando a linguagem Borland Delphi, com ambiente gráfico para o sistema baseado no Microsoft Windows, tecnologias escolhidas pela compatibilidade com o mercado e facilidade de implementação.

Nota-se que todos os projetos analisados compartilham o objetivo de ampliar a visibilidade e o controle que o proprietário possui sobre sua fazenda, por meio do uso de sistemas informatizados. Em sua maioria, as soluções desenvolvidas adotam a arquitetura de aplicação desktop, limitando, assim, a mobilidade e o acesso remoto às informações, fatores cada vez mais importantes no contexto tecnológico atual. Essa constatação reforça a relevância do presente projeto, que busca oferecer uma abordagem mais moderna, acessível e compatível com o uso em campo, especialmente através de dispositivos móveis.