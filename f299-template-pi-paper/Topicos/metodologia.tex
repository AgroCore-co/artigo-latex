O projeto abarcará uma aplicação dedicada a dispositivos móveis, operando em sinergia com uma aplicação web. A funcionalidade central da aplicação móvel consiste em atuar como uma interface especializada para a armazenagem e administração de dados vitais relacionados aos bubalinos. Além disso, ela integra um sistema alimentado por inteligência artificial, destinado a identificar padrões indicativos do período fértil de uma bubalina. Essa abordagem inovadora visa otimizar a gestão de informações cruciais, proporcionando uma ferramenta inteligente para a detecção precisa do ciclo reprodutivo, fundamentando-se na capacidade da IA em reconhecer padrões específicos associados ao estado de fertilidade das bubalinas.

\subsection*{Metodologia de Gerenciamento de Projeto}

Uma ferramenta que desempenhou um papel essencial no desenvolvimento do sistema foi o Figma, amplamente adotado pela equipe para criar o modelo de alta fidelidade do projeto nas plataformas mobile e web. O Figma se destaca como a principal escolha para os designers, oferecendo uma interface colaborativa e intuitiva que permite a criação de interfaces precisas e detalhadas. Além de possibilitar o design visual do sistema, o Figma foi utilizado como um repositório central de informações. Através do FigJam, uma extensão colaborativa da plataforma, a equipe organizou modelagens e diagramas essenciais para o desenvolvimento do projeto, incluindo o diagrama de classes, diagrama de objetos e o diagrama de caso de uso. Esses elementos foram compartilhados em quadros acessíveis a todos os membros da equipe, garantindo que cada participante tivesse acesso às informações críticas para o alinhamento e a continuidade do projeto.

No FigJam, também foram organizados o quadro "\textit{Kanban}" e o diário de bordo da equipe. O "\textit{Kanban}" permitiu uma visualização clara e organizada do fluxo de tarefas, facilitando o acompanhamento das atividades em andamento, pendentes e concluídas, promovendo a transparência e a eficiência na gestão do trabalho. Já o diário de bordo foi uma ferramenta importante para registrar as atividades da equipe, funcionando como um espaço para documentar as tarefas realizadas. Esse registro ajudou a manter todos alinhados e também facilitou o preenchimento do diário de bordo oficial nas etapas seguintes do projeto. Com isso, a equipe conseguiu otimizar o processo de documentação, garantindo que as informações estivessem prontamente disponíveis e organizadas para as fases subsequentes, promovendo maior fluidez no andamento do projeto. Dessa forma, o Figma e o FigJam não só auxiliam no design e na organização do fluxo de trabalho, mas também contribuíram para o gerenciamento eficaz das tarefas e a documentação do progresso do projeto.

Para uma organização mais eficiente e colaborativa, optou-se pela metodologia ágil "\textit{SCRUM}", que oferece uma estrutura flexível e adaptável, essencial para o desenvolvimento contínuo do projeto. Com o "\textit{SCRUM}", a equipe possui uma visão clara e compartilhada do progresso, o que permite ajustes rápidos e alinhamento constante às necessidades emergentes do sistema. O processo começa com o  "\textit{Product Backlog}", uma lista organizada de recursos e funcionalidades que o sistema deve incorporar, organizada em ordem de prioridade. Este "\textit{backlog}" é o guia central do desenvolvimento, sendo revisado e atualizado de acordo com as necessidades do projeto, garantindo que as funcionalidades mais críticas sejam abordadas primeiro. Para uma otimização eficaz do tempo, o "\textit{SCRUM}" emprega ciclos de desenvolvimento denominados "\textit{Sprints}", que são períodos de trabalho definidos com uma data de início e fim, geralmente de duas a quatro semanas. Cada Sprint foca em um conjunto específico de objetivos a serem entregues. Ao final de cada ciclo, é realizada uma revisão dos resultados obtidos e, se necessário, adaptações são feitas no backlog, proporcionando uma abordagem iterativa e incremental que assegura o cumprimento dos requisitos e a melhora contínua do sistema.

Essa metodologia facilita o planejamento e a adaptação contínua, promove uma comunicação mais eficaz entre os membros da equipe, além de assegurar que todos estejam alinhados quanto ao status e às metas do projeto. A implementação do "\textit{SCRUM}" fortalece a eficiência e a qualidade das entregas, contribuindo para o sucesso do projeto e a satisfação das expectativas dos usuários finais.

Para o desenvolvimento do projeto, foi elaborado um modelo de negócios utilizando a plataforma disponibilizada pelo Serviço Brasileiro de Apoio às Micro e Pequenas Empresas (Sebrae). Esta abordagem estratégica busca fornecer uma visão abrangente dos elementos essenciais do empreendimento, incluindo proposta de valor, segmento de clientes, canais de distribuição, fontes de receita, entre outros. A utilização da plataforma do Sebrae oferece recursos e ferramentas especializadas, alinhadas às melhores práticas de negócios, proporcionando assim um alicerce sólido para a construção e aprimoramento contínuo do modelo de negócios associado ao projeto em questão.

Como mencionado anteriormente, a equipe utilizou a metodologia "\textit{Kanban}" para organizar e acompanhar o fluxo de trabalho no projeto. O "\textit{Kanban}" é uma metodologia de gestão que se baseia em princípios visuais para otimizar o fluxo de trabalho. Originado do Sistema Toyota de Produção, o Kanban utiliza cartões visuais ou sinais para representar unidades de trabalho em um quadro visual. Cada cartão representa uma tarefa, e o quadro é dividido em colunas que representam diferentes estágios do processo. Sua capacidade de proporcionar uma visualização clara e transparente do trabalho em andamento permite que toda a equipe acompanhe o fluxo de atividades e identifique possíveis gargalos ou áreas de melhoria. Além disso, o "\textit{Kanban}" enfatiza a limitação do trabalho em progresso, o que ajuda a evitar sobrecargas e a manter um ritmo de trabalho sustentável \cite{Boeg.J}.

A equipe deliberou pela utilização do FigJam para implementar o "\textit{Kanban}" do projeto, em vez de utilizar o GitHub, por ser uma plataforma que oferece as funcionalidades necessárias para gestão visual das tarefas e documentação centralizada.

\subsection*{Prototipação e Diagramação}
Para a elaboração do protótipo deste projeto, adotou-se a ferramenta Figma, uma plataforma dedicada ao desenvolvimento gráfico com ênfase na criação de interfaces e experiências do usuário. O Figma proporciona aos profissionais da área de design a capacidade de esboçar suas ideias e prototipar projetos de forma eficiente. Essa escolha visa otimizar o processo de design, permitindo uma visualização preliminar das características e funcionalidades da aplicação, antes mesmo da implementação, contribuindo assim para um desenvolvimento mais eficaz e alinhado às expectativas do usuário final.
 
Após desenvolver o protótipo do projeto, foi possível desenvolver os diagramas necessários por meio da ferramenta \textit{online} \href{http://www.lucidchart.com/}{Lucidchart}, que é utilizada em ambientes profissionais para a criação de representações visuais de processos, ideias e estruturas, promovendo uma compreensão clara e eficaz de informações complexas. Além disso, o Lucidchart pode ser integrado a outras plataformas de produtividade, tornando-se uma peça valiosa no arsenal de ferramentas para a visualização e a comunicação eficaz em projetos plataforma online de diagramação e visualização que se destaca no auxílio à criação de diagramas, fluxogramas, mapas mentais, organogramas e Unified Modeling Language (UML).

O Diagrama de Caso de Uso (DCU) é uma ferramenta fundamental na Unified Modeling Language (UML), utilizada para representar e visualizar as interações entre um sistema e seus usuários. Sua principal finalidade é modelar como o sistema será utilizado do ponto de vista do usuário, destacando as diversas funcionalidades oferecidas pela aplicação, por meio de: Atores, Casos de Uso, Associações, Include e Extend.

O Diagrama de Classe é uma ferramenta fundamental na Unified Modeling Language (UML), utilizada para mapear de forma clara a estrutura de um projeto ao modelar as classes, seus atributos, operações e as relações entre os objetos \cite{Bezerra.E}. Essa ferramenta facilita a elaboração das classes, que são representadas por três componentes principais: Nome, atributos e métodos. Além disso, os diagramas de classe detalham os relacionamentos entre as classes, incluindo: Associações, Heranças e Agregações. Esses elementos combinados permitem uma visualização clara e concisa da estrutura do sistema, facilitando a comunicação e o entendimento entre os membros da equipe de desenvolvimento.

O Diagrama de Objetos é uma ferramenta fundamental na Unified Modeling Language (UML), utilizada para modelar as instâncias das classes contidas no diagrama de classes \cite{Bezerra.E}. Ele mostra um conjunto de objetos e seus relacionamentos em um momento particular, representando objetos, seus atributos e os vínculos entre eles. Este diagrama é composto por: Objetos, Atributos e Links.

Também foi necessária a modelagem de redes, para a qual a equipe utilizou a plataforma Draw.io. Essa ferramenta permite criar diagramas de rede de forma eficiente e colaborativa, facilitando a visualização da estrutura de rede necessária para a integração dos dispositivos desktop e móveis. No diagrama, estão representados componentes como o provedor de internet, modem, switch, roteador Wi-Fi e os dispositivos, garantindo uma compreensão clara e detalhada da infraestrutura de rede necessária para o projeto.


\subsection*{Tecnologias para o desenvolvimento da Aplicação}
A fim de gerenciar eficientemente a variedade de dados que serão incorporados ao sistema, abrangendo tipos como data, texto, números, entre outros, tornou-se imperativo realizar a modelagem de um banco de dados, contemplando tanto a perspectiva conceitual quanto lógica.

Para essa tarefa, optou-se por utilizar o brModelo, uma ferramenta especializada em modelagem de banco de dados. Desenvolvida para facilitar a criação e manutenção de modelos conceituais e lógicos o brModelo destaca-se como uma ferramenta valiosa, especialmente nas fases iniciais do desenvolvimento de um sistema de informação. Nessas etapas, compreender e representar de maneira clara e organizada a estrutura de dados torna-se crucial para o sucesso do projeto, e o brModelo oferece recursos que simplificam significativamente esse processo, contribuindo para uma implementação mais eficiente e consistente do banco de dados associado ao sistema em desenvolvimento.


Para o armazenamento dos dados gerados pelos usuários na aplicação, foi selecionado o banco de dados MongoDB, um sistema orientado a documentos, que opera no formato não relacional (NoSQL) e oferece alta flexibilidade e escalabilidade, adequando-se bem às necessidades do projeto. O MongoDB armazena os dados em coleções que contêm documentos no formato JSON (JavaScript Object Notation), o que facilita a leitura e oferece uma estrutura de dados que se adapta rapidamente a mudanças de requisitos sem a necessidade de remodelar toda a base. A estrutura hierárquica do MongoDB, baseada em coleções e documentos, organiza os dados de maneira intuitiva e eficiente, permitindo consultas rápidas e uma gestão mais prática. A abordagem não relacional do MongoDB elimina a necessidade de relacionamentos rígidos entre tabelas, dispensando JOINs complexos e resultando em um desempenho otimizado, especialmente em grandes volumes de dados. Assim, a escolha do MongoDB como repositório de dados oferece um alicerce robusto para gerenciar as informações dos usuários, viabilizando uma aplicação com alta disponibilidade, desempenho aprimorado e facilidade de expansão.

Para potencializar o uso do MongoDB, optamos por utilizá-lo na nuvem por meio do MongoDB Atlas, a solução de banco de dados como serviço (DBaaS) oferecida pelo próprio MongoDB. Essa abordagem traz diversas vantagens ao desenvolvimento e à operação da aplicação, como escalabilidade, segurança e facilidade de gestão. O mesmo permite uma configuração rápida e simplificada, além de oferecer alta disponibilidade e backups automáticos. Dessa forma, os dados gerados pela aplicação são armazenados e acessados de maneira segura e confiável, sem a necessidade de uma infraestrutura local de servidores, o que reduz a complexidade de manutenção.

Partindo Para o desenvolvimento da aplicação web, optou-se por utilizar o Node.js, uma plataforma de código aberto que permite a execução de código JavaScript fora do navegador. Essa tecnologia é especialmente indicada para aplicações que requerem alto desempenho e baixa latência, aproveitando a possibilidade de usar JavaScript tanto no lado do cliente quanto no servidor. Essa unificação de linguagem entre front-end e back-end proporciona maior fluidez no desenvolvimento, reduzindo a complexidade e facilitando a integração entre as duas partes da aplicação.

Com o uso do Node.js, foi possível desenvolver uma API (Application Programming Interface) que permite a manipulação de dados do banco de dados por meio de operações de Cadastro, Leitura, Atualizações e Exclusões (CRUD). A API, além de aumentar a eficiência na comunicação entre os componentes do sistema, oferece uma camada de segurança para a aplicação. À medida que o projeto evolui, a API possibilita a integração com outros sistemas e serviços, sejam eles internos ou externos, de forma segura e eficiente. Isso é essencial para garantir a interoperabilidade e facilita a comunicação entre múltiplos dispositivos e plataformas. Além disso, a API inclui mecanismos de controle de acesso, utilizando autenticação e autorização integradas com tokens JWT (JSON Web Tokens). Esse sistema de autenticação assegura que apenas usuários e serviços autorizados possam acessar determinados dados, proporcionando maior segurança e controle sobre a informação compartilhada.

Em conjunto com o Node.js, também será utilizado o Next.js, um framework robusto que permite o desenvolvimento da interface do sistema (front-end) e da lógica de negócios (back-end). Next.js oferece recursos avançados para renderização, otimização de performance e organização de rotas, facilitando tanto o desenvolvimento de uma interface de usuário intuitiva quanto o gerenciamento de operações no servidor. Assim, a combinação de Node.js e Next.js fortalece a estrutura do sistema, tornando-o escalável, seguro e eficiente, promovendo uma experiência de desenvolvimento simplificada e coesa.






