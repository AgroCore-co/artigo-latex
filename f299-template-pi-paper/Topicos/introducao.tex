A primeira introdução de búfalos no Brasil teria ocorrido por volta de 1895, trazidos por condenados foragidos da Guiana Francesa em um barco que aportou na costa norte da Ilha do Marajó. A introdução mais documentada, no entanto, ocorreu por volta de 1902, com uma importação feita por Bertino Lobato de Miranda para sua Fazenda São Joaquim, localizada às margens do rio Ararí, também na Ilha do Marajó. Esses búfalos eram pretos, de procedência italiana \cite{ABCB2016}. Com o passar dos anos, diversas outras importações foram realizadas por criadores do Marajó, do Baixo Amazonas, do Nordeste, do Sul e de Minas Gerais. Atualmente, o rebanho bubalino no Brasil é de aproximadamente 1.672.956 cabeças \cite{IBGE2023}, distribuídas entre diversos estados.

A criação de búfalos é de grande importância para o atendimento da demanda alimentar (leite e carne) e também para a economia, tanto no Brasil quanto no mundo. Esses animais apresentam vantagens em relação a outros ruminantes domésticos, principalmente no que diz respeito à rusticidade, à capacidade de aproveitamento de alimentos de baixa qualidade, à adaptação a terrenos alagadiços e às variadas condições climáticas e de manejo \cite{EMBRAPA2019}.

O leite de búfala possui características únicas que o diferenciam do leite de vaca. Segundo Marques, em seu livro Criação de Búfalos, na elaboração de laticínios, o leite bubalino apresenta um rendimento industrial cerca de 40\% superior ao leite bovino. Além disso, possui 33\% menos colesterol, 48\% mais proteína, 59\% mais cálcio e 47\% mais fósforo. Por conter maior teor de gordura, é necessária uma menor quantidade de leite para a produção de produtos como manteiga e queijos, quando comparado ao leite de vaca \cite{Embrapa1998}.

Com base nessas informações, as búfalas são consideradas excelentes produtoras de leite. Elas podem atingir médias superiores a 7 litros de leite por fêmea/dia, durante lactações de aproximadamente 270 dias. No entanto, a média nacional não ultrapassa 5 litros por fêmea/dia, em lactações de cerca de 250 dias \cite{Embrapa1998}. Para aumentar essa produtividade, práticas como a seleção de matrizes (definindo um mínimo produtivo para permanência no rebanho), a seleção de reprodutores (com foco em valor genético para produção leiteira), o manejo adequado e os cuidados sanitários são essenciais.

Para embasar este projeto, realizaram-se pesquisas de campo em propriedades voltadas à produção de leite de búfala, localizadas na região do Vale do Ribeira. Optou-se pela metodologia de pesquisa qualitativa, com o objetivo de compreender as particularidades de diferentes realidades produtivas. Foram entrevistados dois produtores com perfis distintos, um com um rebanho de 12 cabeças e outro com mais de 400, além de um médico-veterinário especializado, atuante em uma indústria de laticínios da região. Este profissional presta atendimento a diversos fornecedores da empresa, oferecendo uma visão ampla sobre os padrões e cuidados adotados na criação de bubalinos.

A pesquisa revelou que ambos os produtores não utilizam softwares específicos para o manejo de bubalinos e demonstraram interesse em uma solução voltada ao setor, reconhecendo que a adoção de um sistema digital pode tornar a avaliação de rendimento da propriedade mais precisa e reduzir perdas de informações importantes. Durante as entrevistas, foi feita a seguinte pergunta: “Na sua opinião, quais funcionalidades seriam necessárias para que um sistema atendesse à sua forma de trabalho?”. A intenção foi identificar possíveis lacunas na proposta atual do sistema, permitindo o planejamento de futuras atualizações. As respostas indicaram duas sugestões relevantes: (1) a implementação de uma visualização da árvore genealógica dos animais, considerada essencial para o controle e a tomada de decisões estratégicas, como a rotação de piquetes e o dimensionamento de área necessária para alimentação do rebanho; e (2) uma funcionalidade que possibilite identificar rapidamente o animal pelo celular, exibindo seu prontuário com todas as informações disponíveis,  esta última já estava prevista na proposta da equipe de desenvolvimento.

Outro ponto importante observado na entrevista com o médico-veterinário foi sua experiência prévia com softwares voltados ao manejo de bovinos. Embora reconheça a utilidade dessas ferramentas, destacou que há lacunas relevantes ao aplicá-las à bubalinocultura, uma vez que algumas práticas e metodologias diferem. Segundo ele, essa discrepância pode comprometer a precisão dos resultados, ainda que os sistemas bovinos ofereçam uma base inicial útil para avaliação.

O proposto projeto propõe o desenvolvimento de uma aplicação multiplataforma (web e mobile) destinada a auxiliar pequenos, médios e grandes produtores de leite de búfala. A solução permitirá o registro individualizado dos animais do criadouro, possibilitando a geração de métricas que apoiem a tomada de decisões quanto ao rendimento da propriedade. Entre as funcionalidades previstas, destacam-se: prontuário com histórico completo de tratamentos e doenças, apoio ao manejo dos animais e, principalmente, o controle detalhado da lactação — etapa fundamental para o aumento da produtividade. A ferramenta buscará oferecer ao produtor uma visão clara do desempenho do rebanho, contribuindo para a identificação de animais com baixa produção e potencial descarte, bem como para o fortalecimento de estratégias de seleção e reprodução.

Este projeto está alinhado às áreas temáticas definidas pelo Fórum de Pró-Reitores de Extensão das Universidades Públicas Brasileiras (FORPROEX), atendendo aos eixos de Meio Ambiente (5) e Tecnologia (7). Também contempla as linhas de extensão voltadas ao Desenvolvimento de Produtos (7), Desenvolvimento Tecnológico (10), Gestão do Trabalho (22) e Saúde Animal (43). Tal enquadramento reforça o compromisso da proposta com os princípios da extensão universitária, contribuindo para a integração entre conhecimento científico, demandas sociais e inovação prática no setor agropecuário \cite{FORPROEX}.