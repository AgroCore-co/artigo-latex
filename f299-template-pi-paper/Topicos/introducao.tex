A primeira introdução de búfalos no Brasil teria ocorrido por volta de 1895, trazidos por condenados foragidos da Guiana Francesa em um barco que aportou na costa norte da Ilha do Marajó. A introdução mais documentada, no entanto, ocorreu por volta de 1902, com uma importação feita por Bertino Lobato de Miranda para sua Fazenda São Joaquim, localizada às margens do rio Ararí, também na Ilha do Marajó. Esses búfalos eram pretos, de procedência italiana \cite{ABCB2016}. Com o passar dos anos, diversas outras importações foram realizadas por criadores do Marajó, do Baixo Amazonas, do Nordeste, do Sul e de Minas Gerais. Atualmente, o rebanho bubalino no Brasil é de aproximadamente 1.672.956 cabeças \cite{IBGE2023}, distribuídas entre diversos estados.

A criação de búfalos é de grande importância para o atendimento da demanda alimentar (leite e carne) e também para a economia, tanto no Brasil quanto no mundo. Esses animais apresentam vantagens em relação a outros ruminantes domésticos, principalmente no que diz respeito à rusticidade, à capacidade de aproveitamento de alimentos de baixa qualidade, à adaptação a terrenos alagadiços e às variadas condições climáticas e de manejo \cite{EMBRAPA2019}. As búfalas são consideradas excelentes produtoras de leite, elas podem atingir médias superiores a 7 litros de leite por fêmea/dia, durante lactações de aproximadamente 270 dias. No entanto, a média nacional não ultrapassa 5 litros por fêmea/dia, em lactações de cerca de 250 dias \cite{Embrapa1998}. Para aumentar essa produtividade, práticas como a seleção de matrizes (definindo um mínimo produtivo para permanência no rebanho), a seleção de reprodutores (com foco em valor genético para produção leiteira), o manejo adequado e os cuidados sanitários são essenciais.

Apesar da relevância da produção de leite bubalino, ainda não existem sistemas digitais específicos para o manejo desses animais. As soluções atualmente disponíveis são direcionadas ao gado bovino, o que pode gerar inconsistências no controle da reprodução, lactação e produtividade, devido às particularidades fisiológicas dos búfalos. Muitos produtores ainda utilizam planilhas eletrônicas, tornando a consulta de dados históricos e o registro de novas informações um processo massivo, complexo e suscetível a erros devido ao grande volume de dados. Outros dependem de anotações em papel, que podem se perder com a degradação física e dificultam a consulta rápida e confiável. Essa lacuna tecnológica compromete a organização das informações, a análise de desempenho e a tomada de decisão estratégica pelos produtores.

Com o objetivo de compreender como é realizado o manejo em fazendas voltadas à produção de leite de búfala, foram realizadas visitas a propriedades localizadas na região do Vale do Ribeira. Por meio dessas visitas, foi possível conduzir a pesquisa técnica, adotando-se a metodologia qualitativa, a qual permitiu entender as particularidades de diferentes realidades produtivas. Foram entrevistados dois produtores com perfis distintos, um com um rebanho de 12 cabeças e outro com mais de 400, além de um médico-veterinário especializado, atuante em uma indústria de laticínios da região. Esse profissional presta atendimento a diversos fornecedores da empresa, oferecendo uma visão ampla sobre os padrões e cuidados adotados na criação de bubalinos.

A pesquisa com os produtores revelou que ambos não utilizam softwares específicos para o manejo de bubalinos. Ao ser feita a pergunta: “Existe algum software específico utilizado para o gerenciamento do manejo de fazenda com foco na lactação de búfalos? Se sim, qual?”, ambos responderam que desconhecem a existência de um sistema específico para o manejo de bubalinos. Eles mencionaram conhecer plataformas desenvolvidas para o manejo de bovinos, o que não se mostra totalmente efetivo, pois, quando se trata de informações reprodutivas, o tempo de gestação do bovino é de cerca de 9 meses (SILVA, 2020), enquanto o dos bubalinos é de aproximadamente 10 meses (EMBRAPA, 2007). Dessa forma, há uma diferença média de cerca de 30 dias, o que faz com que softwares com valores pré-definidos acusassem que as búfalas estão com a reprodução atrasada.

Seguindo a mesma linha de investigação, foi questionado o interesse em utilizar um sistema específico para auxiliar o manejo: “Se existisse um software específico para o gerenciamento do manejo de fazendas com foco na lactação de búfalos, que possuísse funcionalidades para identificar os búfalos com desempenho abaixo da média e para acompanhar as informações sanitárias, zootécnicas e de lactação, como ele poderia impactar a gestão e melhorar os resultados da propriedade?”. Ambos demonstraram interesse em uma solução voltada ao setor, reconhecendo que a adoção de um sistema digital poderia tornar a avaliação do desempenho da propriedade mais precisa e reduzir perdas de informações importantes.

Durante as entrevistas, também foi feita a pergunta: “Na sua opinião, quais funcionalidades seriam necessárias para que um sistema atendesse à sua forma de trabalho?”. O objetivo foi identificar possíveis lacunas na proposta atual do sistema, permitindo o planejamento de futuras atualizações. As respostas indicaram duas sugestões relevantes: (1) a implementação de uma visualização da árvore genealógica dos animais, considerada essencial para o controle das matrizes presentes na propriedade, visando sempre as que mais produzem leite; (2) uma funcionalidade que possibilite identificar rapidamente o animal por meio do celular, exibindo seu prontuário com todas as informações disponíveis, sendo que esta última já estava prevista na proposta do sistema em desenvolvimento.

Além das propriedades voltadas à produção de leite, também foi realizada uma entrevista com representantes de um Instituto de Zootecnia localizado na mesma região. Durante a entrevista, foi possível observar que o controle do rebanho, no instituto, é realizado por meio de planilhas eletrônicas separadas por áreas temáticas, como pesagem dos animais, produção de leite, registros de cruzamentos, tratamentos e separação de grupos. Cada planilha contém múltiplas abas correspondentes aos anos de registro, exigindo a repetição manual de informações entre diferentes arquivos e períodos. Essa estrutura fragmentada torna o processo massivo e suscetível a erros, além de dificultar a manutenção de um histórico consolidado dos dados do rebanho.

Este projeto está alinhado às áreas temáticas definidas pelo Fórum de Pró-Reitores de Extensão das Universidades Públicas Brasileiras (FORPROEX), atendendo aos eixos de Meio Ambiente (5) e Tecnologia (7). Também contempla as linhas de extensão voltadas ao Desenvolvimento de Produtos (7), Desenvolvimento Tecnológico (10), Gestão do Trabalho (22) e Saúde Animal (43). Tal enquadramento reforça o compromisso da proposta com os princípios da extensão universitária, contribuindo para a integração entre conhecimento científico, demandas sociais e inovação prática no setor agropecuário \cite{FORPROEX}.