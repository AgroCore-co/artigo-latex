Em 2015, a Organização das Nações Unidas (ONU) montou a Agenda 2030, que se trata de um pacto global assinado por 193 países, para que em 2030 exista um mundo melhor para todos os povos e nações. A agenda é composta por 17 Objetivos de Desenvolvimento Sustentável (ODS), sendo estes objetivos ambientais e sociais, que se desenrolam em 169 metas, com o foco em superar os principais desafios de desenvolvimento enfrentados no Brasil e no mundo. O objetivo oito aborda o Trabalho Decente e Crescimento Econômico, já o objetivo nove fala sobre Indústria Inovação e Infraestrutura. O oitavo objetivo tem como propósito promover o crescimento econômico inclusivo e sustentável, para que assim todos atinjam níveis elevados de produtividade por meio da modernização tecnológica e inovações, principalmente em setores de alto valor e grande demanda de mão de obra. Também é de grande relevância mencionar sobre o nona ODS, Indústria, inovação e infraestrutura, que prezam por construir organizações resilientes, promover a industrialização de qualidade, confiável, o emprego pleno e produtivo e o trabalho digno para todos. \cite{onu}.

A criação de bubalinos, ou búfalo, é uma prática milenar que remonta a várias civilizações antigas, são animais domésticos da família dos bovídeos que foram  inicialmente introduzidos, no Brasil, ao final do século XIX pela região Norte em pequenos lotes, de origem asiática, criados principalmente por sua carne, leite, couro e força de trabalho. Sua  seleção e criação seletiva ao longo de milênios levaram ao desenvolvimento de diferentes raças de búfalos asiáticos, cada uma adaptada às condições locais e às necessidades humanas. As raças de bubalinos mais comuns no Brasil são as \textit{Mediterrâneo, Murrah, Jafarabadi} e \textit{Carabao}. Entretanto para a criação dos bubalinos existem algumas restrições, como no controle de índices zootécnico: Idade, peso que o bezerro foi desmamado, Saúde e Bem-Estar, Taxa de Concepção, Peso ao Abate, entre outras.  O controle sanitário, com a assistência da maioria dos rebanhos por um médico veterinário, é de extrema importância para o desenvolvimento dos bubalinos. \cite{Correa.G}
  
Porém, aqueles que possuem criadouros de bubalinos, eventualmente, enfrentam complicações e deficiência na área de gestão, coleta e controle de informações de seus búfalos. Com as dificuldades enfrentadas no gerenciamento de sua criação, pode ocorrer perdas no controle e faturamento de seu negócio.

O Brasil é um dos maiores exportadores de carne bovina do mundo, tornando crucial fornecer informações precisas e completas aos compradores para assegurar uma transação transparente e legal dos animais.  

Com base nas informações apresentadas, nossa equipe planeja desenvolver um sistema integrado para auxiliar na gestão dos criadouros de bubalinos, focando no controle sanitário, reprodutivo e produtivo. A solução proposta será composta por uma aplicação web, que permitirá o acesso centralizado às informações e a integração com dispositivos móveis, possibilitando o gerenciamento remoto em tempo real. O banco de dados armazenará todos os registros de maneira organizada e segura, garantindo a rastreabilidade e o fácil acesso às informações históricas sobre o rebanho, como saúde, reprodução e produtividade.

Além disso, a utilização de Inteligência Artificial (IA) permitirá a análise avançada dos dados coletados. A IA será responsável por identificar padrões e gerar previsões, como alertas para problemas de saúde recorrentes, otimização dos períodos de reprodução, e sugestões de manejo baseadas em dados de desempenho. Com isso, o sistema oferecerá suporte direto à tomada de decisões estratégicas, facilitando a gestão do rebanho e minimizando riscos.

Os principais beneficiados serão os haras de bubalinos, locais especializados na criação e reprodução de búfalos. O sistema garantirá o acompanhamento rigoroso dos padrões de controle sanitário e reprodutivo, assegurando que os búfalos se desenvolvam de forma saudável e produtiva. Além de melhorar o controle dos índices zootécnicos, o projeto contribuirá para a sustentabilidade e eficiência da pecuária bubalina no Brasil.