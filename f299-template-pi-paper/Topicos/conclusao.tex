Com o propósito de contribuir para a consecução dos Objetivos de Desenvolvimento Sustentável estabelecidos pela ONU, destaca-se a relevância da oitava ODS, voltada para o crescimento econômico. Nesse contexto, aprimorar o gerenciamento dos dados relacionados aos bubalinos, incluindo informações zootécnicas, sanitárias e histórico do animal, torna-se crucial. Tal aprimoramento visa mitigar a falta de transparência em situações como a venda dos animais para potenciais clientes.
Adicionalmente, a nona ODS, centrada na inovação e infraestrutura, integra-se harmoniosamente ao escopo do projeto em questão. Este, focado na otimização do manejo de bubalinos por meio de uma aplicação móvel inovadora, destaca-se pela incorporação de tecnologias avançadas, como algoritmos de inteligência artificial. Essa abordagem inovadora visa aprimorar significativamente a eficiência no manejo, proporcionando benefícios tangíveis para a sustentabilidade econômica e ambiental, alinhando-se assim aos objetivos globais delineados pela ONU.

O sistema proposto visa proporcionar aos responsáveis pelo manejo de bubalinos uma plataforma eficaz para o cadastro e monitoramento de seus animais. Através do sistema, será possível realizar o cadastro detalhado de cada búfalo, permitindo a coleta e armazenamento dos dados essenciais em seus respectivos prontuários. Esse processo não apenas simplifica a gestão, mas também promove transparência, oferecendo aos criadores um controle mais preciso sobre as informações relacionadas aos seus rebanhos.
Além disso, o sistema incorpora recursos de inteligência artificial destinados a detectar o período de cio das búfalas. Essa funcionalidade não apenas aprimora a eficiência reprodutiva do criadouro, mas também fornece insights valiosos para otimizar estratégias de reprodução. A capacidade de identificar o momento propício para a reprodução é um avanço significativo no gerenciamento reprodutivo do rebanho.Em síntese, o sistema proposto não apenas simplifica os processos de gerenciamento de bubalinos, mas também integra tecnologias avançadas para aprimorar a eficiência reprodutiva e promover práticas sustentáveis no manejo do rebanho.

Adicionalmente, como parte essencial para futuros desenvolvimentos, identifica-se a necessidade de aprofundar a compreensão no gerenciamento do manejo, permitindo a geração automática de relatórios. Esse aprimoramento possibilitará uma análise mais abrangente e eficaz das informações coletadas, contribuindo para uma tomada de decisão mais informada por parte dos responsáveis pelo manejo de bubalinos.
Outro ponto a ser considerado em perspectiva futura é a contínua análise e aprimoramento da aplicação móvel. Isso incluirá a realização de testes práticos junto a diferentes criadouros, a fim de validar a eficácia da aplicação em variados cenários de manejo. A condução desses testes proporcionará valiosos insights para ajustes e refinamentos contínuos, garantindo que a aplicação atenda plenamente às necessidades e nuances específicas de diversos criadouros. Esse ciclo de aprimoramento contínuo assegura a adaptabilidade e relevância da solução ao longo do tempo.

Em síntese, este estudo destaca a significativa relevância comercial e o potencial benéfico da plataforma para o manejo de bubalinos. Embora haja a necessidade de realização de testes e análises completas para avaliar a viabilidade do projeto, seu embasamento científico sólido e o potencial para aprimorar o gerenciamento do manejo de bubalinos o qualificam como uma iniciativa promissora. Esta plataforma mostra-se capaz de contribuir para o avanço do conhecimento e o desenvolvimento de soluções efetivas no contexto do manejo sustentável de rebanhos bubalinos.
