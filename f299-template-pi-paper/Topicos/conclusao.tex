O desenvolvimento da plataforma demonstrou avanços significativos na organização e registro das informações zootécnicas, sanitárias, reprodutivas e produtivas do rebanho. Ao substituir planilhas e anotações manuais, a aplicação oferece uma interface simplificada e intuitiva, permitindo que os usuários registrem, consultem e analisem dados de maneira ágil e organizada. Essa melhoria reduz erros de registro, aumenta a consistência das informações e facilita o acesso a históricos completos de cada animal.

Embora a separação de funções entre as plataformas web e mobile apresente grande potencial, atualmente a versão mobile possui uma limitação quanto à disponibilidade, estando acessível apenas em dispositivos com sistema operacional Android.

O sistema já foi populado com dados reais fornecidos pelo Instituto de Zootecnia, permitindo validar sua funcionalidade e a integração com o banco de dados principal. No entanto, é necessário realizar testes prolongados em produção, uma vez que a geração contínua e a manutenção dos dados precisam ser avaliadas para estimar os custos de hospedagem e confirmar o desempenho em larga escala.

A inteligência artificial, implementada por meio do modelo Random Forest Regressor, foi utilizada para fornecer predições da produção de leite de cada animal, considerando informações históricas e características individuais. O modelo atingiu um coeficiente de determinação (R²) de 0,73, demonstrando que consegue explicar a maior parte da variabilidade observada na produção de leite. Esse desempenho evidencia que a IA é adequada para uso prático, fornecendo estimativas confiáveis que podem subsidiar decisões de manejo, sem substituir o julgamento do gestor ou veterinário.

Dessa forma, o projeto Buffs apresenta grande potencial para se tornar o primeiro sistema voltado especificamente para o manejo de bubalinos, contribuindo para que propriedades rurais aumentem a eficiência e a produtividade do rebanho.
